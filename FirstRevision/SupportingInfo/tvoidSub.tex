If a calibration measurement with known diffusion coefficient is used the relationship of $R$ can be formulated as used 
as well. $R_D$ stands for the dependency of $R$ the calibration
\begin{equation}
R_D(w) = 6\lambda \left( \coth{\left( \frac{1}{2\lambda} \right) } - 2\lambda \right)
\end{equation}

\begin{equation}
\lambda = \frac{D V}{\Vc w^2} = \frac{D \AL}{\Vc w}
\end{equation}

In order to eliminate $\tvoid$ as experimental input, eq. \ref{eq:CFEquation} is used as a substitution for the 
retention ratio, resulting in an expression solely dependent on $\te$ and $w$:
\begin{equation}
R_{\te}(w) = \frac{2 \CF w}{\te}
\end{equation}
By adjusting $w$ such that \[\left( R_{\te} - R_D \right)^2 \rightarrow \min\]
$w$ can be calculated. This corresponds (as an exact solution) to the literature to the "calibration method 5", 
presented by Wahlund \scite{Wahlund2013} and remains as the only recommended calibration method according to the 
results presented in the main paper.